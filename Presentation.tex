\documentclass{beamer}
\usepackage[utf8]{inputenc}
\usepackage[russian]{babel}
\usepackage[T2A]{fontenc}
\usepackage{amsmath}
\usepackage{amsfonts}
\usepackage{amsthm}
\usepackage{bbm}
\usepackage{amssymb}
\usepackage{bm}
\usepackage{graphicx}
\usepackage{epstopdf}
\usepackage[]{algorithm2e}
\usepackage{amsthm}

%\usetheme{Warsaw}
%\usecolortheme{sidebartab}
\usetheme{Warsaw}
\usecolortheme{seahorse}

%\definecolor{beamer@blendedblue}{RGB}{255,255,0}
%\definecolor{beamer@blendedblue}{HTML}{008A34} %green
%\definecolor{beamer@blendedblue}{HTML}{4A4A4A} %grey
%\definecolor{beamer@blendedblue}{HTML}{0E9059} %biryuz
\definecolor{beamer@blendedblue}{HTML}{027466} %blue


\theoremstyle{definition}
\newtheorem{defin}{Definition}
\newtheorem{assumption}{Assumption}
\theoremstyle{plain}
\newtheorem{thm}{Theorem}
\newtheorem{lem}{Lemma}
\newtheorem{prop}{Proposition}
\theoremstyle{remark}
\newtheorem{remark}{Remark}
\newtheorem{prob}{Problem}
\def\eqdef{\stackrel{def}{=}}

% \DeclareMathOperator*{\argmin}{argmin} 
% \DeclareMathOperator*{\Argmin}{Argmin} 
% \DeclareMathOperator{\barcnt}{bar}
% \DeclareMathOperator{\supp}{supp}
% \DeclareMathOperator{\est}{\mathbb{E}}
% \DeclareMathOperator{\inter}{int}
% \DeclareMathOperator{\ind}{Ind}

\begin{document}
\setlength{\abovedisplayskip}{5pt}
\setlength{\belowdisplayskip}{5pt}

	\title[\hbox to 60mm{High-dimensional integrals \hfill\insertframenumber\,/\,15}]
			{ Course project \\ <<High-dimensional integrals for option pricing>>}
	\author[A. Podkopaev, N. Puchkin, I. Silin]{\large Alexander Podkopaev, Nikita Puchkin, Igor Silin}
	\institute[Affiiation]{
	\textsc{Skolkovo institute of science and technology} 
	}

\date{\footnotesize{December 16, 2016}}

	\begin{frame}
		\titlepage
	\end{frame}

	\begin{frame}{Plan}
		  \tableofcontents[
		    sectionstyle=show/show,
		    subsectionstyle=show/show/show
		  ]
	\end{frame}
	
	\section{Introduction to option pricing }
		\begin{frame}{Introduction to option pricing}
		 
		\end{frame}

	\section{From Black-Scholes to diffusion }

		\begin{frame}{Black-Scholes equation}
			\vspace{-5pt}
			\begin{block}{Black-Scholes equation}
				\begin{equation}
					\begin{aligned}
						&\frac{\partial c(s,t)}{\partial t} + 
						rs \frac{\partial c(s,t)}{\partial s} + 
						\frac12 \gamma^2 s^2 \frac{\partial^2 c(s,t)}{\partial s^2} = 
						rc(s,t),\\
						&c(s,T^{\prime}) = g(s), \;\;\;\;\; s \in \mathbbm{R}_{+},\\
						&c(0,t) = 0, \;\;\;\;\;\;\;\;\;\;\;\; t \in [0;\;T^{\prime}].
						\nonumber
					\end{aligned}
				\end{equation}
			\end{block}

			\vspace{-5pt}
			\begin{block}{Substitution}
						\begin{itemize}
							\item New variable: $x=\ln s$
							\item New initial condition: $f(x) = e^{\frac{1}{\gamma^2} (r-\frac{\gamma^2}{2})x} g(e^x)$, 
							\item New coefficients: $\sigma = \frac12 \gamma^2$,
						 			$V(x,t) = V = r +  \frac{1}{2\gamma^2} \left( r-\frac{\gamma^2}{2}\right)^2$,
						 	%\item New soluton: $c(s,t) = e^{-\frac{1}{\gamma^2} (r-\frac{\gamma^2}{2})\ln s} u(\ln s, T^{\prime} - t)$.
						 	\item New solution: $u(x,t) =  e^{\frac{1}{\gamma^2} (r-\frac{\gamma^2}{2})x} c(e^x, T^{\prime} - t)$
						\end{itemize}
			\end{block}
							
		\end{frame}

		\begin{frame}{Diffusion equation}

			\begin{block}{One-dimensional reaction-diffusion equation}
				\begin{equation}
					\begin{aligned}
						&\frac{\partial u(x,t)}{\partial t} = \sigma \frac{\partial^2 u(x,t)}{\partial x^2} - V(x,t) u(x,t),
						\;\;\; t \in [0;\;T^{\prime}],\\
						&u(x,0) = f(x), \;\;\; x \in \mathbbm{R}.
					\nonumber
					\end{aligned}
				\end{equation}
			\end{block}

			\begin{center}
				Fast method for solving this equation was proposed in the paper:\\
				''A low-rank approach to the computation of path integrals'',\\ M. Litsarev, I. Oseledets, 2015.
			\end{center}
		\end{frame}

	\section{Idea of the method}
		\begin{frame}{Idea of the method}
		 
		\end{frame}
		

	\section{Experiments}
		\begin{frame}{Experiments: european put option}
		 
		\end{frame}

		\begin{frame}{Experiments: something else (bounded!)}
		 
		\end{frame}

	\section{Discussion}

		\begin{frame}{Discussion}
			\vspace{-7pt}
			\begin{block}{Unbounded terminal condition}
				\begin{itemize}
					\item If $n$ is big, numerical solution $\rightarrow +\infty$.\\
					Explanation: \\substitution $x = \ln s$ and one speciality of the algorithm.
					\item If $n$ is small, numerical solution is not accurate and cross approximation is not useful.
				\end{itemize}
			\end{block}
			\vspace{-3pt}
			\begin{block}{Bounded terminal condition}
				\begin{itemize}
					\item No such failure with big $n$ as in the previous case.
					\item The bigger $n$, the more accurate solution.
					\item Cross approximation is useful: allows to reduce complexity.
				\end{itemize}
			\end{block}
			\centering{More detailed analysis on that in our report.}
		\end{frame}
	
	\section*{}

		\begin{frame}{}
			\centering{ \LARGE Thanks for your attention!}
		\end{frame}

\end{document}